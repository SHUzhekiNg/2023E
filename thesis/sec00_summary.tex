\MinParskip{}

Light pollution is affecting our environment, biodiversity, biological clock, etc. Regarding this issue, we mainly need to solve three problems in our paper: establish a model to measure the light pollution, propose policies to reduce light pollution and evaluate the effectiveness of our policies. 

First of all, we analyze the light pollution, and attribute it to night-light intensity. Then we select several indicators which are relevant to night-light intensity more or less, and got data from official web or database.

In order to build a model that can be applied to access the light pollution in all locations, we used selected indicators to reflect the night-light intensity to evaluate the light pollution. After determining the index system, we use the improved Entropy Weight Method(EWM) to calculate the weights of indicators. Then, we use Technique for Order of Preference by Similarity to Ideal Solution(TOPSIS) to calculate the score of night-light intensity. We apply our model to four types of locations: urban communities, suburban communities, rural communities and protected land locations. In the application to 12 selected states, results are all in line with the reality.

Combining the results of our model and the reality of light pollution, we try to propose policies to reduce light pollution. Different policies varies greatly in different types of locations, so we take four types of locations into consideration and make three strategies in different aspects. The first one aims at technology to improve the efficiency of lighting, the second one aims at governor’s regulations and the last one aims at improving residents’ awareness of light pollution. Our policies aim to adjust selected indicators data, so it’s operational and intuitive. 

To evaluate the effectiveness of our policies, we select two positions: California and New York to implement our strategies. California is categorized as a suburban community and New York is categorized as an urban community. After exploring local situations, we made several tailored strategies relevant to the policies we addressed before. With our tailored strategy implementation, light pollution in both of them will be alleviated to some extend as shown in our model results.

Finally, we carry out a sensitivity analysis of the evaluation model, demonstrating the robustness of our models. We also write a flyer to the community officials and local groups in California to promote the strategy for the locals.


\textbf{Keywords:} optimized Entropy Weight Method, TOPSIS, Light Pollution.