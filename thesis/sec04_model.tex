%% part4
\MinParskip{}

\section{The EWM-TOPSIS Model}

\subsection{Indicators of the Model}
We selected 9 cities as evaluation objects. As the night-time light intensity directly reflects the extent of light pollution, so it is regarded as the direct evaluation indicator, and we chose 8 indirect indicators as evaluation indicator variables to evaluate the night light level of the city.

The eight indicators we selected are population density, night-time electricity consumption, per capita GDP, regional industrial structure, regional average working hours, the last bus time of public transport, night life index, limiting magnitude, etc. Among these indicators, only the limiting magnitude is an objective indicator and cannot be changed.

To explain the selection of indicators, We divide the indicators into two categories. One is the benefit-attributed indicators. The larger the data index, the higher the night-time light intensity. The other is the cost-attributed indicators. The smaller the data index, the higher the night-time light intensity.

For the benefit-attributed indicators, there are 5 indicators which are listed below.

/*NOTDONENOTDONE
\begin{enumerate}
    \item Population density. With the increase of population density in a region, the lighting demand of its public space also increases, thus increasing the night light level, so it is a benefit data. 
 
    \item Night-time electricity consumption. A large part of nighttime electricity consumption is used for lighting, so it can also reflect the nighttime light level to a certain extent, which is a benefit data. 
     
    \item GDP per capita. GDP per capita reflects the degree of local development, the more developed and prosperous the region, the more shopping malls and high-rise buildings in prosperous areas, the LED lights in shopping malls and the glass curtain walls of high-rise buildings will lead to an increase in nighttime light levels, which is a benefit data. 
     
    \item Regional industrial structure. the demand for lighting in various industries is different, the primary industry has little demand for light, some industries in the secondary industry have requirements for light, and the tertiary industry service industry is extremely dependent on light to attract customers, so the industrial structure will also have an impact on the local nighttime light level, so we believe that the more the proportion of the tertiary industry, the higher the nighttime light level, which is a benefit data. 
     
    \item Nightlife index. Nightlife means living the same life at night as during the day, so the lighting demand increases. The higher the nightlife index of a region, the higher the nighttime light level, which belongs to the benefit data.
    
    \item The last bus time of public transport. The operation guarantee condition of public transportation is lighting, so the lighting will naturally increase during the traffic operation time. The later the last bus is, the higher the night lighting level will be, which belongs to the benefit data.
\end{enumerate}

For the cost-attributed indicators, there are 2 indicators which are listed below.
\begin{enumerate}
    \item The average working hours in the region. The earlier people leave work, the earlier the end of the evening commute. So the demand for high-beam lights will become less, and the demand for night lighting time will become less, so it will affect the level of night lighting, so the longer the working hours, the later the work, which belongs to the cost data.
    
    \item Limit magnitude. Refers to the darkest magnitude that can be observed through a telescope. The higher the limit magnitude that can be seen, the lower the brightness on the ground, that is, the low level of illumination at night, so it belongs to the cost data.
\end{enumerate}




NOTDONENOTDONE*/

\subsection{The Entropy Weight Method (EWM) }
We use the Entropy Weight Method (EWM) to get the weight of indicators. The Entropy Weight Method is an object weighting method, which uses the information entropy idea for reference. It is used to determine the relative weights of different criteria in a decision-making process. It determines the weight of the indicators by calculating the information entropy of the index, according to the impact of the relative change of the index on the overall system.

In other words, the weight of each indicator is given according to the difference degree of the indicator value, and the corresponding weight of each indicator is obtained. The indicator with relatively large change degree has larger weight. 

The greater the entropy, the more chaotic the system is, the less information it carries and the less weight it has. Otherwise, the higher the entropy, the more orderly the system is, the more information it carries, and the greater the weight is.

The first step of the Entropy Weight Method is the standardization of measured values. The standardized value of the jth index in the ith sample is denoted as $p_{ij}$, and its calculation method is as follows:$$p_{ij}=\frac{a_{ij}}{\sum_{i=1}^na_{ij}},i=1,2\cdots,,9,j=1,2,\cdots,8$$

Secondly, in the EWM, the entropy value $e_i$ of the jth index is defined as:$$e_j=-\frac{1}{lnn}\sum_{i=1}^np_{ij}lnp_{ij},j=1,2,\cdots,8$$
The range of entropy value $e_j$ is [0, 1]. The larger the $e_j$ is, the greater the differentiation degree of index j is, and more information can be derived. Hence, higher weight should be given to the index. 

Finally, the calculation method of weight is:$$w_j=\frac{1-e_j}{\sum_{j=1}^me_j},j=1,2,\cdots,8$$
Through the three steps, we can easily get the weight of each indicator.

\subsection{The TOPSIS Method}
The Technique for Order of Preference by Similarity to Ideal Solution (TOPSIS) is a multi-criteria decision-making method that is used to determine the best option from a set of alternatives based on multiple criteria. 

Since the selected indicators are benefit-based data, which means the bigger the data, the better, so we use Min-Max normalization to process the data.For benefit-based data,$b_{ij}=\frac{a_{ij}-a_j^{min}}{a_j^{max}-a_j^{min}}$ ,and for cost-based data,$b_{ij}=\frac{a_{j}^{max}-a_{ij}}{a_j^{max}-a_j^{min}}$. All these data from the normalized matrix, and it is called B.

We then multiply each value in a column with the corresponding weight given.

Calculating Ideal Best and Ideal worst. Now we need to calculate Euclidean distance for elements in all rows from the ideal best and ideal worst, Here $s_i^*$ is the best distance calculated on an *ith* row, where $b_{ij}$ is element value and $C_i^{*}$ is the ideal best for that column. similarly, we can find $s_i^0$, worst distance calculated on an *ith* row.

// insert a table

Calculating Topsis Score and Ranking. Now we have distance positive and distance negative with us, let’s calculate the Topsis score for each row based on them.

\subsection{Application of the Model on the Four Locations}
\subsubsection{A Protected Land Location: }
\subsubsection{A Rural Community: }
\subsubsection{A Suburban Community: }
\subsubsection{An Urban Community: }