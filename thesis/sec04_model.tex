%% part4
\MinParskip{}

\section{The EWM-TOPSIS Model}

\subsection{Indicators of the Model}
We selected 9 cities as evaluation objects. As the night-time light intensity directly reflects the extent of light pollution, so it is regarded as the direct evaluation indicator, and we chose 8 indirect indicators as evaluation indicator variables to evaluate the night light level of the city.
\begin{enumerate}
    \item 
\end{enumerate}

\subsection{The Entropy Weight Method (EWM) }
We use the entropy weight method (EWM) to get the weight of indicators. The entropy method is an object weighting method, which uses the information entropy idea for reference. It is used to determine the relative weights of different criteria in a decision-making process. It determines the weight of the indicators by calculating the information entropy of the index, according to the impact of the relative change of the index on the overall system.

In other words, the weight of each indicator is given according to the difference degree of the indicator value, and the corresponding weight of each indicator is obtained. The indicator with relatively large change degree has larger weight. 

The greater the entropy, the more chaotic the system is, the less information it carries and the less weight it has; The higher the entropy, the more orderly the system is, the more information it carries, and the greater the weight is.

\subsection{The TOPSIS Method}
The Technique for Order of Preference by Similarity to Ideal Solution (TOPSIS) is a multi-criteria decision-making method that is used to determine the best option from a set of alternatives based on multiple criteria. 


