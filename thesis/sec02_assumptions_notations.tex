%% part2
\MinParskip{}
\section{Assumptions and Notations}
\subsection{Assumptions and Justifications} 

\begin{itemize}
    \item Assume that 
\end{itemize}

\textbf{Justifications: }

\begin{itemize}
    \item Assume that 
\end{itemize}

\textbf{Justifications: }


\subsection{Notations}
The mostly used symbols and their definitions are defined (Table 1), and more symbols will be defined later in the text. 
\begin{table}[h] \centering
    \caption{Notations of symbols}
    \begin{tabular}{ccl}
        \toprule
        Symbol & Description\\ \hline
        $a_{ij}$ & The value of the j-th indicator of the i-th city to be evaluated. \\
        $A=(a_{ij})_{i×j}$ & Data matrix composed by $a_{ij}$. \\
        $p_{ij}$ & The proportion of the i-th evaluation objective to the j-th indicator. \\
        $e_j$ & The entropy value of the j-th indicator. \\
        $w_j$ & The weight of the j-th indicator. \\
        $b_{ij}$ & The data from $a_{ij}$ after standard 0-1 transformation.\\
        $B=(b_{ij})_{i×j}$ & The decision matrix composed by $b_{ij}$ after processing.\\
        $C=W*B$ & The vector normalization attribute matrix after weighting. \\
        $C_i$ & The i-th roll of the $C$ matrix. \\
        $C^*=[b_1^{max},b_2^{max},\cdots,b_j^{max}]$ & The positive ideal solution. \\
        $C^*=[b_1^{min},b_2^{min},\cdots,b_j^{min}]$ & The negative ideal solution. \\
        $s_i^*$ & The distance from $C_i$ to the positive ideal solution. \\
        $s_i^0$ & The distance from $C_i$ to the negative ideal solution. \\
        $f_i$ & The night-time light intensity of the i-th city. \\
        \bottomrule
    \end{tabular}
\end{table}