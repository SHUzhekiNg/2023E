% part5
\MinParskip{}

\section{Strategy Making and Implementation}



\subsection{Strategies for Addressing Light Pollution}

\subsubsection{Avoid blue lights at night}
Blue-rich white light sources are also known to increase glare and compromise human vision, especially in the aging eye. These lights create potential road safety problems for motorists and pedestrians alike. In natural settings, blue light at night has been shown to adversely affect wildlife behavior and reproduction, particularly in cities, which are often stopover points for migratory species.

Outdoor lighting with strong blue content is likely to worsen skyglow because it has a significantly larger geographic reach than lighting consisting of less blue. \cite{light_pollution}

\subsubsection{Strategy 2}

\subsubsection{Protecting Dark-Sky Areas}
Dark-sky areas are becomming less and less because of human development ......

\begin{itemize}
    \item Establish and designate dark-sky areas that are free of artificial light pollution.

    \item Implement and enforce regulations that prohibit outdoor lighting in designated dark-sky areas.

    \item Encourage the use of low-intensity lighting in dark-sky areas, such as amber or red lights, to minimize light pollution while still providing visibility.

    \item Conduct public outreach and education campaigns to promote awareness of the importance of dark-sky areas and the negative impacts of light pollution on wildlife and human health.

    \item Work with community leaders, landowners, and stakeholders to establish dark-sky areas and develop sustainable tourism opportunities around them..
\end{itemize}


\subsection{Tailored Strategy Implementation}

\subsubsection{Location 1}

\subsubsection{Location 2}


