% part5
\MinParskip{}

\section{Strategy Making and Implementation}



\subsection{Strategies for Addressing Light Pollution}

\subsubsection{Avoid blue lights at night}
Blue-rich white light sources are also known to increase glare and compromise human vision, especially in the aging eye. These lights create potential road safety problems for motorists and pedestrians alike. In natural settings, blue light at night has been shown to adversely affect wildlife behavior and reproduction, particularly in cities, which are often stopover points for migratory species.

Outdoor lighting with strong blue content is likely to worsen skyglow because it has a significantly larger geographic reach than lighting consisting of less blue. \cite{light_pollution}

\subsubsection{Strategy 2}

\subsubsection{Strategy 3}


\subsection{Tailored Strategy Implementation}

\subsubsection{Location 1}

\subsubsection{Location 2}


