% part5
\MinParskip{}

\section{Strategy Making and Implementation}

\subsection{Strategies for Addressing Light Pollution}

According to the problem 3, to address light pollution, we designed three possible intervention startegies as follows.

\subsubsection{Lighting Design and Technology}

Lighting design and technology play a crucial role in reducing light pollution. Poor lighting design and the use of outdated lighting products can result in excessive glare, sky glow, and light trespass, which can disrupt ecosystems and interfere with human sleep cycles. Our goal is to promote research and development of new lighting technologies that are designed to minimize light pollution while still meeting lighting needs. Here are some actions that can be taken into account.

\begin{enumerate}
    
    \item \textbf{Develop and implement smart lighting technologies that reduce light pollution and save energy.}
    \begin{itemize}
        \item \textbf{Install sensors that adjust lighting levels based on natural lighting conditions, such as sunlight and moonlight. }
        \item \textbf{Install motion sensors and timers to reduce the amount of unnecessary outdoor lighting.}
        
        \item \textbf{Develop and deploy smart city technologies that optimize lighting efficiency and minimize light pollution in public spaces.}
    \end{itemize}

    \item \textbf{Use directional lighting that illuminates only the areas that need to be lit, rather than casting light in all directions.}

    Directional lighting can be used in landscape lighting to highlight specific features, such as trees or statues, while minimizing light pollution and unwanted glare. This creates a more visually appealing and sustainable lighting solution for outdoor spaces.

    \begin{itemize}
        \item \textbf{Use lighting fixtures that direct light downward, reducing sky glow and light trespass.}
        \item \textbf{Use shielding to direct light only where it is needed.}
    \end{itemize}

    \item \textbf{Avoid blue-rich white light sources at night. }

    Blue-rich white light sources are also known to increase glare and compromise human vision, especially in the aging eye. These lights create potential road safety problems for motorists and pedestrians alike.

\end{enumerate}

These actions not only help protect the environment, but also save energy and reduce the cost of outdoor lighting.

We hope these regulations and policies can make influence to these indicators:
\begin{itemize}
    \item \textbf{Power Consumption per Capita per Month}. Using the state-of-art lighting designs and technologies, power consumption can be reduced for sure.
    \item \textbf{All tertiary industry percentage}. The emphasis on lighting design and technology researches can bring up the secondary industry percentage while slightly reduce the percentage of tertiary industry.
\end{itemize}

\subsubsection{Government Regulations and Policies}
Government regulations and policies can help reduce light pollution in a more compulsory way, and lead a environment-friendly social trend. Among the policies listed below, the dark-sky-areas-related plans take a big part, and also are easier and better for government to realize the idea to some extent.
\begin{enumerate}
    \item \textbf{Establish lighting standards that require businesses and homeowners to use lighting fixtures that minimize light pollution.}
    
    \item \textbf{Develop and enforce outdoor lighting regulations that reduce glare, sky glow, and light trespass.}
    
    \item \textbf{Encourage the use of energy-efficient lighting through tax credits and rebates.}

    \item \textbf{Establish and designate dark-sky areas that are free of artificial light pollution.} 
    
    Dark-sky areas are becomming less and less because of human expansion. The dark-sky-area plan can be regarded as one of the most practical and considerable plans in consideration of human development and urbanization. 
    
    There are many actions can be taken for the dark-sky area plan. For example, 
    \begin{itemize}
        \item \textbf{Implement and enforce regulations that prohibit outdoor lighting in designated dark-sky areas.}
        \item \textbf{Encourage the use of low-intensity lighting in dark-sky areas to minimize light pollution while still providing visibility.}
        \item \textbf{Work with community leaders, landowners, local organizations,and stakeholders to establish dark-sky areas and develop sustainable tourism opportunities around them.}
    \end{itemize}
\end{enumerate}

We hope these regulations and policies can make influence to these indicators:
\begin{itemize}
    \item \textbf{Limiting Magnitude}. As time goes on, with the regulations and policies taken place, the limit magnitude can certainly witness a significant improvement.
    \item \textbf{Population Density}. With more and more dark-sky areas open to the public, we hop it will promote rural area travelling, while decreasing population density.
    \item \textbf{Working hours per Week}. Same as the population density, people can possibly have more time of leisure and enjoy the beautiful night sky.
\end{itemize}

\subsubsection{Promoting Responsible Lighting Practices}
Promoting responsible lighting practices raises awareness about the negative impacts of excessive lighting and encourages individuals and organizations to take action and develop awarenesses of sustainable action series. Through these actions, we hope it can help create more sustainable and livable communities for all.
\begin{enumerate}
    \item \textbf{Implement lighting ordinances and regulations for outdoor lighting to reduce unnecessary and excessive lighting.}
    
    \item \textbf{Promote the use of energy-efficient lighting technologies such as LED lights that reduce energy consumption and light pollution.}
    
    \item \textbf{Encourage homeowners and businesses to turn off non-essential lighting during late-night hours to reduce light pollution.}
    
    \item \textbf{Propagandize and educate the public on responsible lighting practices and the negative impacts of light pollution, and develop awarenesses of action.}
\end{enumerate}

We hope these regulations and policies can make influence to these indicators:
\begin{itemize}
    \item \textbf{Nightlife index} We hope the consciousness of action can be turned into less night-time illumination for nightlife. People can spend more time in less illuminated activities at home or outside.
    \item \textbf{Working hours per Week} Work more at daylight time and reduce the working hours in general.
    \item \textbf{Last Bus Time} As the working hours have been moved up, we expect the last bus time can be moved up as well.
\end{itemize}





\subsection{Tailored Strategy Implementation}
We chose California and New York for the following tailored strategy making, for California represents the biggest suburban area and New York represents the metropolis. 
\subsubsection{California: Government Regulations and Policies}



\subsubsection{New York: Technology Innovations and Practice Promotions}


