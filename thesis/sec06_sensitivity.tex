%%part6
\MinParskip{}

\section{Sensitivity Analysis}

We chose California to test the sensitivity of our model. We selected two indicators, regional industrial structure and limit magnitude, which are cost-based and benefit-based, to test the sensitivity of our model.

1. Regional industrial structure

California's regional industrial structure is 0.2229. We have studied the change of California's light-light intensity when the regional industrial structure changes from 0.20 to 0.25. From the \textbf{\textit{chart}}, we can see that the night-light intensity of California is 0.52997 at this time. When the regional industrial structure changes to 0.20, the night-light intensity decreases to whatwhatwhatwhat, and the relative change rate is. The relative change rate from 0.2229 to 0.20 is whatwhatwhatwhat, and the ratio of the two relative change rates is. When the regional industrial structure changes to 0.25, the night-light intensity increases to whatwhatwhatwhat, and the relative change rate is whatwhatwhatwhat. The relative change rate from 0.2229 to 0.20 is whatwhatwhatwhat, and the ratio of the two relative change rates is whatwhatwhatwhat. Therefore, we can see that our model will not deviate greatly due to the deviation of the regional industrial structure.

2. Limit magnitude

California's limit magnitude is 6. We have studied the change of California's light-light intensity when the limit magnitude changes from 5 to 7. From the chart, we can see that the night-light intensity of California is 0.52997 at this time. When the limit magnitude changes to 5, the night-light intensity decreases to whatwhatwhatwhat, and the relative change rate is whatwhatwhatwhat. The relative change rate from 0.2229 to 5 is whatwhatwhatwhat, and the ratio of the two relative change rates is whatwhatwhatwhat. When the limit magnitude changes to 7, the night-light intensity increases to whatwhatwhatwhat, and the relative change rate is whatwhatwhatwhat. The relative change rate from 0.2229 to 7 is whatwhatwhatwhat, and the ratio of the two relative change rates is whatwhatwhatwhat. Therefore, we can see that our model will not have a large deviation due to the deviation of Limit Magnitude.