%%part7
\MinParskip{}

\section{Strengths and Weaknesses}
\subsection{Strengths}

Our work aims at investigating trends of global language users and their distrubution situation. With this model, we put forward targeted proposal for a multinational service company, and optimize its planned number of offices. At last, we evaluate the effectiveness of our model. To sum up the above, the model and the policies proposed have the following strengths:

\begin{itemize}
    \item Inclusive
\end{itemize}

The model involves 5 indicators, well presenting most of the major factors determining the trends of golbal languages and their distribution. This makes the data analysis and policy-making reliable and rigorous.

\begin{itemize}
    \item Quantified
\end{itemize}


\begin{itemize}
    \item Comparitive
\end{itemize}


\subsection{Weaknessess}

Despite the advantages, there are still some shortcomings in our models and the proposal:

\begin{itemize}
    \item Ignoring External Shocks
\end{itemize}

As a major premise, we exclude catastrophic disasters and wars. Too big these changes are that we cannot precisely predict the trend of world population distribution and language development trends afterward. But in real world, all circumstances are possible, so our model is still limited.

\begin{itemize}
    \item Ignoring the Second Generation of the Immigrants
\end{itemize}

We generally consider that native languages will not change after immigration to another cultural circle. But after 50 years, the next generation of the immigrants are born. Our model does not take their native languages into account.

