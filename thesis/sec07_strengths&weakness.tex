%%part7
\MinParskip{}

\section{Strengths and Weaknesses}
\subsection{Strengths}

\begin{itemize}
    \item The selection of evaluation indicators is scientific and comprehensive. We get the data from The U.S. Bureau of Economic Analysis and other scientific databases. We also make necessary modifications to better address our purpose, including involving the education context and adjust specific indicators.
    
    \item It's convenient for governor to propose strategy based on our model. Since our indicators are specific and operational, governors can propose targeted strategies. 
    
    \item We used optimized EWM to calculate the weight instead of traditional EWM, which overcomes shortcoming of unrealistic weights.

    \item Comprehensive application of multiple methods. We use relatively objective methods - EWM to identify the weights of the multiple indicators. Then, we use TOPSIS to calculate night-light intensity of each location. The combination of multiple methods has constructed a scientific evaluation system.
    
    \item The evaluation model is robust. We did a sensitivity analysis on the night-light intensity. We made several indicators fluctuate in a small range and found that the list of locations' night-light intensity did not change significantly, indicating that our evaluation model has high stability.
    Weaknesses:
\end{itemize}


\subsection{Weaknessess}
\begin{itemize}
    \item The model does not consider differences between light colors. Based on several researches of light pollution, specific use of light color can reduce light pollution in some extend.
    
    \item We didn't get the specific data of how much consumption is on lighting, so we use another index to instead. But the indicator of Power Consumption per Capita per Month just reflects residents' usage on lighting in some extend.
\end{itemize}